\documentclass[10pt,a4paper]{article}
%
\usepackage{amsmath}
\usepackage{amssymb}
\usepackage{graphicx}
\usepackage{texnames,path}
%\usepackage[german]{babel}

% \usepackage{aaspp4}
% \usepackage[light,first,bottomafter]{draftcopy}
%
\def\BE{\mbox{\boldmath $E$}}
% \def\BBE{\mbox{\boldmath $\bar{E}$}}

\begin{document}
\pagestyle{headings}

\title{deltasql - A Simple but Powerful Synchronization Mechanism for Databases}

\author{Tiziano Mengotti\thanks{tiziano.mengotti@gmail.com}\\
deltasql.sourceforge.net\\}

\maketitle
\thispagestyle{empty}

\begin{abstract}
This paper describes a simple but powerful algorithm to synchronize databases. It assumes that each change to the database can be scripted in SQL. To each script, a version number is attached. A special table named tbsynchronization remembers the last version number applied to the database. When a database is updated, the last version number is retrieved, and the algorithm generates the newer scripts which still need to be applied. The paper describes in detail how the algorithm works when development on a main line splits in several branches. It describes several other ideas, which make deltasql a unique synchronization system for databases. 
\end{abstract}


\section{Introduction}
\label{sec:intro}
This document is intended for developers, who would like to experiment with the advanced features of the GPU\footnote{Global Processing Unit or Gnutella Processing Unit} network. Newcomers first should take a look at [3,4,5] for an introduction to the GPU framework.
The framework extensions described here were made possible by the VisualSynapse [1] and Synapse [2] libraries. These libraries provide Delphi components for \texttt{UDP}, \texttt{TCP}, \texttt{HTTP} and several other protocols.



\section{Conclusion}

Copyright \copyright  2007-2013 the deltasql Development Team, all rights reserved.


\newpage

\section{References}

\vspace{0.2cm}
\vspace{0.2cm}
\noindent
[1] R. Tegel,  {\em VisualSynapse}, 2004, available from \path|http://visualsynapse.sourceforge.net|.

\vspace{0.2cm}
\noindent
[2] L. Gebauer,  {\em Synapse}, 2004, available from \path|http://synapse.ararat.cz|.


\end{document}


